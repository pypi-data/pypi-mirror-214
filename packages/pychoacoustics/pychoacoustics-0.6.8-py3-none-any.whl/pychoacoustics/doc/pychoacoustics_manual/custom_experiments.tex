\chapter{Designing Custom Experiments}
\label{sec:cutom_exp}

In order to add a new experiment to \texttt{pychoacoustics}, create a directory in your home folder called \texttt{pychoacoustics\_experiments}, inside this folder create a subfolder called \texttt{custom\_experiments}. Each experiment is written in a single file contained in this folder. Let's imagine we want to create an experiment for a simple frequency discrimination task. We create a file named \verb+freq.py+ in the \texttt{custom\_experiments} folder. In addition to the experiment file we need an additional file that lists all the experiments contained in the \texttt{custom\_experiments} directory. This file must be named \verb+__init__.py+, and in our case it will have the following content:
\begin{lstlisting}[stepnumber=0]
__all__ = ["freq"]
\end{lstlisting}


here the variable \verb+__all__+ is simply a python list with the name of the experiment files. So, if one day we decide to write a new experiment on, let's say, level discrimination, in a file called \verb+lev.py+ we would simply add it to the list in \verb+__init__.py+:
\begin{lstlisting}[stepnumber=0]
__all__ = ["freq",
           "lev"]
\end{lstlisting}
For people familiar with packaging Python modules it should be clear by now that the custom experiments folder is basically a Python package containing various modules (the experiment files). If at some point we want to remove an experiment from \texttt{pychoacoustics}, for example because it contains a bug that does not allow the program to start, we can simply remove it from the list in \verb+__init__.py+

Let's go back to the \verb+freq.py+ file. Here we need to define four functions. For our example the names of these functions would be:
\begin{lstlisting}[stepnumber=0]
initialize_freq()
select_default_parameters_freq()
get_fields_to_hide_freq()
doTrial_freq()
\end{lstlisting}
basically the function names consist of a fixed prefix, followed by the name of the experiment file. So in the case of the level experiment example written in the file \verb+lev.py+, our four functions would be called:
\begin{lstlisting}[stepnumber=0]
initialize_lev()
select_default_parameters_lev()
get_fields_to_hide_lev()
doTrial_lev()
\end{lstlisting}
we'll look at each function in details shortly. Briefly, the \verb+initialize_+ function is used to set some general parameters and options for our experiment; the \verb+select_default_parameters_+ function lists all the widgets (text fields and choosers) of our experiment and their default values; the \verb+get_field_to_hide_+ function is used to dinamically hide or show certain widgets depending on the status of other widgets; finally, the \verb+doTrial_+ function contains the code that generates the sounds and plays them during the experiment.

\subsubsection{The \texttt{initialize\_} function}
The \verb+initialize_+ function of our frequency discrimination experiment looks like this:
\begin{lstlisting}
def initialize_freq(prm):
  exp_name = "Frequency Discrimination Demo"
  prm["experimentsChoices"].append(exp_name)
  prm[exp_name] = {}
  prm[exp_name]["paradigmChoices"] = ["Adaptive",
                                      "Weighted Up/Down",
                                      "Constant m-Intervals n-Alternatives"]

  prm[exp_name]["opts"] = ["hasISIBox", "hasAlternativesChooser", 
                           "hasFeedback", "hasIntervalLights"]
    
  prm[exp_name]["execString"] = "freq"
  return prm
\end{lstlisting}
When the function is called, it is passed a dictionary containing various parameters through the ``prm'' argument.
The function receives this dictionary of parameters and adds or modifies some of them.
In the first line we give a label to the experiment, this can be anything we want, except the label of an experiment already existing.
The second line adds this experiment label to the list of ``experimentsChoices''.
The third line creates a new sub-dictionary that has as a key the experiment label.
Next we list the paradims that our experiment supports by creating a ``paradigmChoices'' key and giving
the names of the supported paradigms as a list. These paradims listed here must be within the set of paradims 
supported by \texttt{pychoacoustics} (see Section~\ref{sec:paradigms} for a description of the paradigms currently supported).
In the next line we set an ``\texttt{opts}'' key containing a list of options. The full list of options that can be set here
is described in details in Section~\ref{sec:experiment_opts}. In brief, for our experiment we want to have a widget to set the ISI between presentation
intervals (\texttt{hasISIBox}), a widget to choose the number of response alternatives (\texttt{hasAlternativesChooser}),
a widget to set the feedback on or off for a given block of trials (\texttt{hasFeedback}), and finally we want lights
to mark the observation intervals (\texttt{hasIntervalLights}).
The penultimate line of the \verb+initialize_+ function sets the ``\texttt{execString}'' of our experiment. This
must be the name of our experiment file, so in our case ``\texttt{freq}''.



\subsubsection{The \texttt{select\_default\_parameters\_} function}
The \texttt{select\_default\_parameters\_} function is the function in which you
define all the widgets (text fields and choosers) needed for your experiment.
For our frequency discrimination experiment, the first lines look as follow:
\begin{lstlisting}
def select_default_parameters_freq(parent, paradigm, par):
   
  field = []
  fieldLabel = []
  chooser = []
  chooserLabel = []
  chooserOptions = []
\end{lstlisting}
the function accepts three arguments, ``parent'' is simply a reference to the pychoacoustics
application. ``paradigm'' is the paradigm with which the
function has been called, while ``par'' is a variable
that can hold some special values for initializing the function. The use of the ``par'' argument is discussed in Section~\ref{sec:par}.

From line three to line seven, we create a series of empty lists. The \texttt{field} and \texttt{fieldLabel } lists will
hold the default values of our text field widgets, and their labels, respectively. The \texttt{chooser} and \texttt{chooserLabel}
lists will likewise hold the default values of our chooser widgets, and their labels, while the \texttt{chooserOptions} list will hold 
the possible values that our choosers can take. Lines 8 to 29 show how we populate these lists for our frequency discrimination
experiment:

\begin{lstlisting}[firstnumber=8]
  fieldLabel.append("Frequency (Hz)")
  field.append(1000)

  fieldLabel.append("Starting Difference (%)")
  field.append(20)
    
  fieldLabel.append("Level (dB SPL)")
  field.append(50)
   
  fieldLabel.append("Duration (ms)")
  field.append(180)
    
  fieldLabel.append("Ramps (ms)")
  field.append(10)

    
  chooserOptions.append(["Right",
                         "Left",
                         "Both"])
  chooserLabel.append("Ear:")
  chooser.append("Right")
\end{lstlisting}

The last lines of our \texttt{select\_default\_parameters\_} function are used to set some additional parameters and look as follows:
\begin{lstlisting}[firstnumber=29]
  prm = {}
  if paradigm == None:
      prm['paradigm'] = "Adaptive"
  else:
      prm['paradigm'] = paradigm
  prm['adType'] =  "Geometric"
  prm['field'] = field
  prm['fieldLabel'] = fieldLabel
  prm['chooser'] = chooser
  prm['chooserLabel'] = chooserLabel
  prm['chooserOptions'] =  chooserOptions
  prm['nIntervals'] = 2
  prm['nAlternatives'] = 2

  return prm
\end{lstlisting}
on line 29 we create a dictionary to hold the parameters. Lines 30--33 are used to
set a default paradigm for our experiment if \texttt{None} has been passed to our function.
\texttt{'adType} gives sets the default type of the adaptive procedure, this could be either \texttt{Geometric}, or 
\texttt{Arithmetic}. From line 25 to line 39 we insert in the dictionary the \texttt{field}, \texttt{fieldLabel},
\texttt{chooser}, \texttt{chooserLabel} and \texttt{chooserOptions} lists that we have previously creaetd and populated.
Finally in the last two lines we give the default number of response intervals and response alternatives.

\subsubsection{The \texttt{get\_fields\_to\_hide\_} function}

The purpose of the \verb+get_fields_to_hide_+ function is to dinamically show or
hide certain widgets depending on the status of other widgets. This function must
be defined, but is not essential to a \texttt{pychoacoustics} experiment, so if you want to read
all the essential information first, you can simply write the following:
\begin{lstlisting}
def get_fields_to_hide_freq(parent):
  pass
\end{lstlisting}
and move on to read about the next function, otherwise, read on.

Let's suppose that you 
want to set up a frequency discrimination experiment in which the frequency of the 
standard stimulus may be either fixed, or change from trial to trial. You start by writing
an experiment with a single ``Frequency'' text field for the fixed stimulus frequency case.
You then add two additional fields called ``Min. Frequency'' and ``Max Frequency'' to set
the range of frequencies in the roving frequency case. Finally, you create a chooser to decide
whether an experiment is to be run with a fixed or roving frequency. The code for creating these widgets
is shown below:



\subsubsection{The \texttt{doTrial\_} function}

\subsubsection{The Experiment ``opts''}
\label{sec:experiment_opts}

\begin{itemize}
\item \verb+hasISIBox+
\item \verb+hasAlternativesChooser+
\item \verb+hasFeedback+
\item \verb+hasIntervalLights+
\item \verb+hasPreTrialInterval+
\end{itemize}

\subsubsection{Using \texttt{par}}
\label{sec:par}



\section{Simulations}
\label{sec:simulations}

\texttt{pychoacoustics} is not designed to run simulations in itself, however it provides a hook
to redirect the control flow to an auditory model that you need to specify yourself in the experiment file.

You can retrieve the current response mode from the experiment file with:
\begin{lstlisting}[stepnumber=0]
parent.prm['allBlocks']['responseMode']
\end{lstlisting}
so, in the experiment file, after the creation of the stimuli for the trial you can redirect the
control flow of the program depending on the response mode:
\begin{lstlisting}
if parent.prm['allBlocks']['responseMode'] != "Simulated Listener":
   #we are not in simulation mode, play the stimuli for the listener
   parent.playSoundSequence(sndSeq, ISIs)
if parent.prm['allBlocks']['responseMode'] == "Simulated Listener":
   #we are in simulation mode
   #pass the stimuli to an auditory model and decision device
   #---
   #Here you specify your model, pychoacoustics doesn't do it for you!
   # at the end your simulated listener arrives to a response that is
   # either correct or incorrect
   #---
   parent.prm['trialRunning'] = False 
   #this is needed for technical reasons (if the 'trialRunning'
   #flag were set to 'True' pychoacoustics would not process
   #the response.
   #
   #let's suppose that at the end of the simulation you store the
   #response in a variable called 'resp', that can take as values 
   #either the string 'Correct' or the string 'Incorrect'.
   #You can then proceed to let pychoacoustics process the response:
   #
   if resp == 'Correct':
      parent.sortResponse(parent.correctButton) 
   elif resp == 'Incorrect':
      #list all the possible 'incorrect' buttons
      inc_buttons = numpy.delete(numpy.arange(
                                 self.prm['nAlternatives'])+1, 
                                 self.correctButton-1))
      #choose one of the incorrect buttons
      parent.sortResponse(random.choice(inc_buttons))

\end{lstlisting}
%%% Local Variables: 
%%% mode: latex
%%% TeX-master: "pychoacoustics_manual"
%%% End: 

